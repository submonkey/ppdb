\chapter{\babDua}
\section{\Database}
\label{sec:database}

\Database~yang digunakan dalam sistem PPDB Surabaya 2013 adalah sistem basis
data \f{open source}, yaitu: MySQL Server. Sedangkan versi \database~yang
digunakan adalah:

\begin{enumerate}[noitemsep, nolistsep]
  \item \mysql~5.1.69 (pada sisi \master) dengan menggunakan distro CentOS 6.4.
  \item \mysql~5.1.49.3 (pada sisi \slave) dengan menggunakan distro Debian 6.
\end{enumerate}

% section database (end)

\section{Desain ERD \Database}
\label{sec:desain_erd}

Pada bagian ini akan dipaparkan mengenai desain ERD (\f{Entity Relathionship
Diagram}) \database. Desain diagram tidak akan ditampilkan secara lengkap,
tetapi hanya beberapa tabel utama. Untuk desain diagram selengkapnya dapat
dilihat pada berkas \f{diagram.mwb} (dalam bentuk format MySQL Workbench).
Tabel utama itu antara lain:

\begin{enumerate}[noitemsep, nolistsep]
  \item Tabel yang menyimpan data \master~dari siswa.
  \item Tabel yang menyimpan data pendaftaran dari siswa, dalam hal ini dibagi
    lagi menjadi pendaftaran jalur sekolah kawasan dan pendaftaran jalur umum.
  \item Tabel yang menyimpan hasil seleksi pendaftaran.
\end{enumerate}


% section desain_erd (end)

\section{Topologi \Database}
\label{sec:topologi_database}

Topologi \database~bersifat bertingkat dan menggunakan sistem replikasi dari
sisi \master~ke sisi \slave, lebih jelas mengenai topologi dapat dilihat pada
Gambar~\ref{fig:topologi}. Data dari \master~akan direplikasi oleh \slave,
sehingga keduanya menjadi identik. Pada umumnya pada sisi \master~hanya
mempunyai akses \W, sedangkan pada sisi \slave~hanya mempunyai akses \R. Pada
\bo{Master 1} dan \bo{Master 2}, selain menjadi \master~juga bertindak sebagai
\slave~(\master-\slave). Data yang masuk pada \bo{Master 1} akan disinkronisasi
dengan \bo{Master 2} begitupun sebaliknya.  Antara \master~dan \slave, tidak
selalu dalam satu jaringan atau subnet. Untuk menangani beban jaringan pada
kondisi \master~dan \slave~yang berbeda jaringan, dibutuhkan sejenis penghubung
antar keduanya. Penghubung (\bo{Bridge 1} dan \bo{Bridge 2}) ini bersifat
\master-\slave, bertindak sebagai \slave~dari level di atasnya dan sebagai
\master~bagi level di bawahnya.

\begin{figure}
  \centering
  \begin{tikzpicture}[font=\footnotesize, node distance=2em, auto]
    \node[block] (master1) {Master 1};
    \node[block, right=6em of master1] (master2) {Master 2};
    \node[cloud, below of=master1] (bridge1) {Bridge 1};
    \node[cloud, below of=master2] (bridge2) {Bridge 2};
    \node[cloud, below of=bridge1] (slave12) {Slave 2};
    \node[cloud, left=0.5em of slave12] (slave11) {Slave 1};
    \node[cloud, right=0.5em of slave12] (slave13) {Slave 3};
    \node[cloud, below of=bridge2] (slave22) {Slave 2};
    \node[cloud, left=0.5em of slave22] (slave21) {Slave 1};
    \node[cloud, right=0.5em of slave22] (slave23) {Slave 3};

    \path[arrow] (master1) -- (master2);
    \path[arrow] (master2) -- (master1);
    \path[arrow] (master1) -- (bridge1);
    \path[arrow] (master2) -- (bridge2);
    \path[arrow] (bridge1) -- (slave11);
    \path[arrow] (bridge1) -- (slave12);
    \path[arrow] (bridge1) -- (slave13);
    \path[arrow] (bridge2) -- (slave21);
    \path[arrow] (bridge2) -- (slave22);
    \path[arrow] (bridge2) -- (slave23);
  \end{tikzpicture}
  \caption{Topologi \database~secara garis besar}
  \label{fig:topologi}
\end{figure}

% section topologi_database (end)

\section{Pengolahan Data}
\label{sec:pengolahan_data}

Pengolahan data menjadi faktor utama dalam proses bisnis sistem PPDB Realtime
ini. Kesalahan pengolahan akan menyebabkan informasi menjadi invalid. Dalam
pengolahan data ini, dibutuhkan adanya kakas bantu untuk memudahkan proses ini.
Beberapa kakas bantu yang digunakan oleh penulis yaitu:

\begin{enumerate} [noitemsep, nolistsep]
  \item Libreoffice Calc, kakas bantu ini digunakan untuk membuka berkas \master,
    maupun data yang akan di-\import ke \database~(.mdf,~.xls,.csv) dan mengolah 
    data.
  \item Vim, kakas bantu ini digunakan untuk mengolah data dan membuat \script.
  \item Beberapa CLI dari linux: mysql, scp, ssh dan lain-lain.
\end{enumerate}


% section pengolahan_data (end)
