\chapter{\babTiga}
\section{Perangkingan}
\label{sec:perangkingan}

Proses bisnis yang utama dalam sistem PPDB \f{online} adalah pendaftaran dan
perangkingan. Tetapi dalam bab ini hanya akan dibahas mengenai proses
perangkingan.  Hasil perangkingan ini akan menentukan peserta didik diterima
atau tidak sesuai dengan pilihan sekolahnya. Pada setiap jenjang (SMP, SMA dan
SMK), mempunyai ketentuan yang berbeda-beda baik untuk pilihan maupun
perangkingan. Secara lebih jelas, hal ini dapat dilihat pada
Tabel~\ref{tab:kawasan} dan Tabel~\ref{tab:umum}.

\begin{table}
\centering 
\caption{Jalur sekolah kawasan (PPDB 2013)}
\label{tab:kawasan}
\begin{tabular} {l p{10cm}}
  \hline
  \multicolumn{1}{c}{No} & \multicolumn{1}{c}{Ketentuan} \\
  \hline
  \hline

  1. & Calon peserta didik dapat memilih satu pilihan sesuai wilayah sekolah
  asal dan satu sekolah di dalam atau di luar wilayah sekolah asal, dengan
  urutan pilihan sekolah secara bebas. \\
  2. & Perangkingan menggunakan nilai Total (40\% nilai UN dan 60\% nilai TPA). \\
  3. & Terdapat persyaratan nilai UN minimal. \\
  4. & Prioritas apabila terjadi kesamaan nilai Total adalah nilai Bahasa
  Indonesia, Matematika, IPA dan Bahasa Inggris (khusus untuk SMA) selanjutnya
  waktu pendaftaran. \\

  \hline  
\end{tabular}
\end{table}

\begin{table}
\centering
\caption{Jalur umum (PPDB 2013)}
\label{tab:umum}
\begin{tabular} {l p{10cm}}
  \hline
  \multicolumn{1}{c}{No.} & \multicolumn{1}{c}{Ketentuan} \\
  \hline
  \hline
  1. & Calon peserta didik SMP atau SMA dapat memilih sekolah pada satu subrayon dalam
  wilayah dan satu sekolah di luar subrayon, dengan urutan pilihan sekolah
  secara bebas. \\
  2. & Calon peserta didik SMK dapat memilih satu SMKN dengan pilihan dua kompetensi keahlian. \\
  3. & Prioritas apabila terjadi kesamaan nilai UN adalah Bahasa Indonesia,
  Matematika, IPA, Bahasa Inggris (khusus SMA \& SMK), subrayon kemudian
  wilayah (khusus SMP \& SMA) selanjutnya waktu pendaftaran. \\
  \hline
\end{tabular}
\end{table}

Pada PPDB 2013 ini terdapat tiga jenis jalur penerimaan peserta didik, yaitu
jalur penerimaan \f{offline} (prestasi, mitra warga dan satu lokasi), jalur
sekolah kawasan dan jalur umum. Setiap jalur ini mempunyai prioritas
perangkingan yang berbeda, dalam bab ini hanya akan dibahas mengenai
perangkingan sekolah kawasan dan jalur umum. Untuk jalur penerimaan
\f{offline}, perangkingan masih bersifat sederhana dan dapat dilakukan secara
manual.

Selain ada pembagian jalur penerimaan, terdapat juga pembagian kategori peserta
didik (lihat Tabel~\ref{tab:kategori}). Kategorisasi peserta didik ini
berpengaruh pada penentuan kuota minimum.  Secara sederhana, pagu untuk tiap
sekolah dibagi menjadi tiga jenis, yaitu: pagu umum, pagu rekomendasi dalam
kota dan pagu rekomendasi luar kota. Khusus untuk pagu umum, tidak ada batasan
kuota 1\%.

\begin{table}
\centering 
\caption{Kategori peserta didik}
\label{tab:kategori}
\begin{tabular} {c c l}
  \hline
  Status KK & Status Tamatan & \multicolumn{1}{c}{Kategori} \\
  \hline
  \hline
  DK & DK & Dalam Kota \\
  DK & LK & Mutasi \\
  LK & DK & Rekomendasi Dalam Kota\textsuperscript{(*)} \\
  LK & LK & Rekomendasi Luar Kota\textsuperscript{(*)} \\
  \hline  
  \multicolumn{3}{l}{\footnotesize{(*) Terkena kuota pagu maksimal 1\% dari pagu umum}} \\
\end{tabular}
\end{table}

% section perangkingan (end)

\section{Algoritma Perangkingan}
\label{sec:algoritma_perangkingan}

Dalam proses perangkingan PPDB sebenarnya tidak dilakukan pengurutan data
secara keseluruhan. Setiap peserta didik yang mendaftar akan dilakukan proses
formulasi nilai sesuai dengan ketentuan perangkingan berdasarkan jalur
penerimaan. Nilai formula inilah yang kemudian dibandingkan dengan batas nilai
terbawah dari sekolah yang menjadi pilihan. Apabila nilai formula dan pagu
mencukupi, maka pendaftar itu akan diterima di sekolah tersebut. Jika tidak,
maka akan beralih ke pilihan sekolah selanjutnya atau jika merupakan pilihan
sekolah terakhir, peserta didik dinyatakan tidak diterima. Nilai formula
bersifat \f{unique}, tidak mungkin ada peserta didik yang mempunyai nilai
formula yang sama.

Bagaimana data dapat terurut tanpa adanya proses pengurutan data secara
menyeluruh? Hal ini dimungkinkan dikarenakan ada proses \f{order by} dari nilai
formula yang telah diberikan kepada masing-masing pendaftar ketika ditampilkan
pada halaman web.  Kunci dari efisiensi proses ini adalah meminimalkan proses
\f{order} pada \database.  Secara tidak langsung, ketika data akan ditampilkan
terjadi proses pengurutan.  Dengan metode ini proses rangking dapat
diminimalisir, dikarenakan pengurutan data hanya terjadi saat ditampilkan pada
halaman web.

% section algoritma_perangkingan (end)

\section{Pemenuhan Pagu}
\label{sec:pemenuhan_pagu}

Pemenuhan pagu adalah pagu pendaftaran yang tidak diisi oleh calon peserta
didik baru yang dinyatakan diterima, sampai batas waktu daftar ulang yang telah
ditetapkan. Mekanisme untuk pengisian pagu berbeda berdasarkan jenjang dan
jalur penerimaan. Tetapi secara garis besar mekanisme ini dapat disederhanakan
menjadi:

\begin{enumerate} [noitemsep, nolistsep]
  \item Pemenuhan pagu jalur mitra warga diambil dari calon peserta didik baru
    dibawah passing grade yang telah ditetapkan dalam satu kecamatan.
  \item Pemenuhan pagu jalur sekolah kawasan terbatas pada satu wilayah sekolah
    asal dan mengabaikan pilihan di luar wilayah.
  \item Pemenuhan pagu jalur umum terbatas pada satu subrayon sekolah dan
    mengabaikan pilihan di luar subrayon.
  \item Khusus untuk jenjang SMK, pemenuhan pagu terbatas pada satu sekolah
    dikarenakan tidak ada rayonisasi.
\end{enumerate}

% section pemenuhan_pagu

\section{Algoritma Pemenuhan Pagu}
\label{sec:algoritma_pemenuhan_pagu}

Pemenuhan pagu sebenarnya adalah proses perangkingan kembali tetapi dengan
memasukkan beberapa kriteria seperti yang sudah dijelaskan di atas. Hanya
pilihan yang sesuai dengan kriteria yang akan dilakukan proses perangkingan.
Hasil seleksi perangkingan sebelumnya setelah dikurangi calon peserta didik
yang tidak daftar ulang (disediakan aplikasi untuk ceklist daftar ulang) akan
diolah lagi. Jadi bagi calon peserta didik baru yang tidak daftar ulang, tidak
akan diproses dalam pemenuhan pagu.

Hal yang membedakan dengan perangkingan sebelumnya adalah urutan pilihan yang
bersifat \f{decreement}. Pilihan akan diseleksi dimulai dari pilihan terbesar,
khusus untuk calon peserta didik baru yang diterima di pilihan pertama tidak
perlu diproses. Proses pemenuhan pagu juga melibatkan calon peserta didik baru
yang sama sekali tidak diterima. Jadi secara sederhana pemenuhan pagu adalah
proses perangkingan dengan  urutan pilihan terbalik dan mengabaikan calon
peserta didik yang diterima di pilihan pertama.

Akan tetapi algoritma ini masih belum sempurna, ada beberapa kondisi yang masih
belum dapat ditangani. Beberapa kendala tersebut adalah:

\begin{enumerate} [noitemsep, nolistsep]
  \item Belum dapat menangani kondisi \f{grade} sekolah yang sama dalam satu
    subrayon.
  \item Calon peserta didik yang sama sekali tidak diterima pada seleksi
    sebelumnya, ada kemungkinan diterima bukan pada prioritas pilihannya.
    Khusus kriteria ini sebaiknya tidak menggunakan algoritma perangkingan
    pilihan terbaik.
  \item Belum dapat menangani calon peserta didik baru yang berstatus
    rekomendasi dalam kota maupun luar kota.
\end{enumerate}

% section algoritma_pemenuhan_pagu
