\chapter{\babSatu}
\section{Sekilas Mengenai PPDB}

Penerimaan Peserta Didik Baru, yang disingkat PPDB adalah penerimaan peserta
didik pada Sekolah Dasar (SD), Sekolah Menengah Pertama (SMP), Sekolah Menengah
Atas (SMA) dan Sekolah Menengah Kejuruan (SMK). Dalam proses PPDB \f{online}
2013 hanya menangani penerimaan peserta didik sekolah negeri saja. PPDB
\f{online} 2013 ini pihak ITS (Institut Teknologi Sepuluh Nopember) hanya
bertanggung jawab untuk jenjang SMP, SMA dan SMK saja. PPDB dilakukan setahun
sekali setiap tahun ajaran baru. Yang menarik dan menjadi tantangan dari PPDB
ini adalah, ketentuan yang selalu berubah di tiap tahunnya. 

Pada PPDB 2013 terdapat dua jenis cara pendaftaran, yaitu \f{offline} dan
\f{online}.  Pendaftaran \f{offline} meliputi jalur mitra warga, satu lokasi,
inklusi dan prestasi.  Sedangkan pendaftaran \f{online} meliputi jalur
sekolah kawasan dan jalur umum. Pendaftaran \f{offline} dapat dilaksanakan
di sekolah negeri tujuan, sedangkan untuk pendaftaran \f{online} dapat
dilakukan di mana saja dengan fasilitas internet.

Jalur mitra warga adalah jalur yang dikhususkan bagi calon peserta didik yang
berasal dari keluarga kurang mampu berkartu keluarga Kota Surabaya. Sedangkan
jalur inklusi diperuntukkan bagi calon peserta didik warga kota Surabaya
berkebutuhan khusus yang dapat mengikuti proses pembelajaran secara reguler,
berasal dari sekolah inklusi dan atau surat keterangan dari psikolog. Untuk
jalur prestasi dibedakan menjadi jalur prestasi olahraga, akademis dan
non-akademis. Jalur satu lokasi dikhususkan bagi calon peserta didik baru yang
berasal dari lulusan Sekolah Dasar Negeri yang satu lokasi dengan Sekolah
Menengah Pertama Negeri yang dituju.  

